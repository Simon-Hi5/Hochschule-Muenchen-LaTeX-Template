\chapter{Implementierung}
\label{chap:implementation}

\section{Formeln}

Man kann Formeln wie folgt einfügen: Das Dateiverarbeitungsproblem kann wie folgt modelliert werden. Eine Datei $F$ der Größe $S$ Bytes, aufgeteilt in Blöcke mit maximaler Größe $c$, ist:
\[ F = \langle B_1, B_2, \ldots, B_m \rangle, \qquad m = \left\lceil \frac{S}{c} \right\rceil \]

\section{Algorithmen}

Man kann Algorithmen mithilfe von Pseudocode wie folgt darstellen:

\begin{algorithm}[htbp]
    \caption{Blockbasierte Dateiverarbeitung mit Hashing.}
    \label{algo:filehash}
    \begin{algorithmic}[1]
        \Require Datei $F$, Blockgröße $c$
        \Ensure Datensatz $\langle size, block\_count, sha256 \rangle$

        \State $h \gets \textsc{InitSHA256}()$ \Comment{Initialisiere Digest.}
        \State $size \gets \textsc{Length}(F)$
        \State $block\_count \gets 0$
        \Statex

        \ForAll{$block \in \textsc{Read}(F, c)$}
            \State $\textsc{Update}(h, block)$
            \State $block\_count \gets block\_count + 1$
        \EndFor
        \Statex

        \State $sha256 \gets \textsc{Finalize}(h)$
        \State \Return $\langle size, block\_count, sha256 \rangle$
    \end{algorithmic}
\end{algorithm}

Man kann auf Algorithmen wie folgt verweisen: Algorithmus~\ref{algo:filehash} beschreibt einen blockbasierten Dateiverarbeitungsalgorithmus.

\section{Listen und Aufzählungen}

Man kann Listen und Aufzählungen wie folgt erstellen:

\begin{enumerate}
  \item Initialisiere einen SHA-256-Digest und lese die Dateigröße.
  \item Lese die Datei in Blöcken von maximal $c$ Bytes.
  \item Aktualisiere für jeden Block den Digest und erhöhe den Blockzähler.
  \item Finalisiere den Digest, um den Hash zu erhalten.
  \item Gib $\langle size,\ block\_count,\ sha256 \rangle$ zurück.
\end{enumerate}

\subsection{Codebeispiele}

Man kann Quellcode einfügen, zum Beispiel Python-Code, wie folgt:

\begin{lstlisting}[float=htb, caption={Quellcode des Dateiverarbeitungsalgorithmus.}, label={lst:filehash}]
import hashlib
from pathlib import Path

def process_file(path: str, chunk_size: int = 1 << 20):
    p = Path(path)
    h = hashlib.sha256()
    with p.open("rb") as f:
        while True:
            b = f.read(chunk_size)
            if not b:
                break
            h.update(b)
    return {
        "file": p.name,
        "size_bytes": p.stat().st_size,
        "sha256": h.hexdigest()
    }

if __name__ == "__main__":
    print(process_file("example.dat"))
\end{lstlisting}
